% Options for packages loaded elsewhere
\PassOptionsToPackage{unicode}{hyperref}
\PassOptionsToPackage{hyphens}{url}
%
\documentclass[
]{article}
\usepackage{lmodern}
\usepackage{amssymb,amsmath}
\usepackage{ifxetex,ifluatex}
\ifnum 0\ifxetex 1\fi\ifluatex 1\fi=0 % if pdftex
  \usepackage[T1]{fontenc}
  \usepackage[utf8]{inputenc}
  \usepackage{textcomp} % provide euro and other symbols
\else % if luatex or xetex
  \usepackage{unicode-math}
  \defaultfontfeatures{Scale=MatchLowercase}
  \defaultfontfeatures[\rmfamily]{Ligatures=TeX,Scale=1}
\fi
% Use upquote if available, for straight quotes in verbatim environments
\IfFileExists{upquote.sty}{\usepackage{upquote}}{}
\IfFileExists{microtype.sty}{% use microtype if available
  \usepackage[]{microtype}
  \UseMicrotypeSet[protrusion]{basicmath} % disable protrusion for tt fonts
}{}
\makeatletter
\@ifundefined{KOMAClassName}{% if non-KOMA class
  \IfFileExists{parskip.sty}{%
    \usepackage{parskip}
  }{% else
    \setlength{\parindent}{0pt}
    \setlength{\parskip}{6pt plus 2pt minus 1pt}}
}{% if KOMA class
  \KOMAoptions{parskip=half}}
\makeatother
\usepackage{xcolor}
\IfFileExists{xurl.sty}{\usepackage{xurl}}{} % add URL line breaks if available
\IfFileExists{bookmark.sty}{\usepackage{bookmark}}{\usepackage{hyperref}}
\hypersetup{
  pdftitle={STAT461 HW6},
  pdfauthor={Xiangyu Ren},
  hidelinks,
  pdfcreator={LaTeX via pandoc}}
\urlstyle{same} % disable monospaced font for URLs
\usepackage[margin=1in]{geometry}
\usepackage{color}
\usepackage{fancyvrb}
\newcommand{\VerbBar}{|}
\newcommand{\VERB}{\Verb[commandchars=\\\{\}]}
\DefineVerbatimEnvironment{Highlighting}{Verbatim}{commandchars=\\\{\}}
% Add ',fontsize=\small' for more characters per line
\usepackage{framed}
\definecolor{shadecolor}{RGB}{248,248,248}
\newenvironment{Shaded}{\begin{snugshade}}{\end{snugshade}}
\newcommand{\AlertTok}[1]{\textcolor[rgb]{0.94,0.16,0.16}{#1}}
\newcommand{\AnnotationTok}[1]{\textcolor[rgb]{0.56,0.35,0.01}{\textbf{\textit{#1}}}}
\newcommand{\AttributeTok}[1]{\textcolor[rgb]{0.77,0.63,0.00}{#1}}
\newcommand{\BaseNTok}[1]{\textcolor[rgb]{0.00,0.00,0.81}{#1}}
\newcommand{\BuiltInTok}[1]{#1}
\newcommand{\CharTok}[1]{\textcolor[rgb]{0.31,0.60,0.02}{#1}}
\newcommand{\CommentTok}[1]{\textcolor[rgb]{0.56,0.35,0.01}{\textit{#1}}}
\newcommand{\CommentVarTok}[1]{\textcolor[rgb]{0.56,0.35,0.01}{\textbf{\textit{#1}}}}
\newcommand{\ConstantTok}[1]{\textcolor[rgb]{0.00,0.00,0.00}{#1}}
\newcommand{\ControlFlowTok}[1]{\textcolor[rgb]{0.13,0.29,0.53}{\textbf{#1}}}
\newcommand{\DataTypeTok}[1]{\textcolor[rgb]{0.13,0.29,0.53}{#1}}
\newcommand{\DecValTok}[1]{\textcolor[rgb]{0.00,0.00,0.81}{#1}}
\newcommand{\DocumentationTok}[1]{\textcolor[rgb]{0.56,0.35,0.01}{\textbf{\textit{#1}}}}
\newcommand{\ErrorTok}[1]{\textcolor[rgb]{0.64,0.00,0.00}{\textbf{#1}}}
\newcommand{\ExtensionTok}[1]{#1}
\newcommand{\FloatTok}[1]{\textcolor[rgb]{0.00,0.00,0.81}{#1}}
\newcommand{\FunctionTok}[1]{\textcolor[rgb]{0.00,0.00,0.00}{#1}}
\newcommand{\ImportTok}[1]{#1}
\newcommand{\InformationTok}[1]{\textcolor[rgb]{0.56,0.35,0.01}{\textbf{\textit{#1}}}}
\newcommand{\KeywordTok}[1]{\textcolor[rgb]{0.13,0.29,0.53}{\textbf{#1}}}
\newcommand{\NormalTok}[1]{#1}
\newcommand{\OperatorTok}[1]{\textcolor[rgb]{0.81,0.36,0.00}{\textbf{#1}}}
\newcommand{\OtherTok}[1]{\textcolor[rgb]{0.56,0.35,0.01}{#1}}
\newcommand{\PreprocessorTok}[1]{\textcolor[rgb]{0.56,0.35,0.01}{\textit{#1}}}
\newcommand{\RegionMarkerTok}[1]{#1}
\newcommand{\SpecialCharTok}[1]{\textcolor[rgb]{0.00,0.00,0.00}{#1}}
\newcommand{\SpecialStringTok}[1]{\textcolor[rgb]{0.31,0.60,0.02}{#1}}
\newcommand{\StringTok}[1]{\textcolor[rgb]{0.31,0.60,0.02}{#1}}
\newcommand{\VariableTok}[1]{\textcolor[rgb]{0.00,0.00,0.00}{#1}}
\newcommand{\VerbatimStringTok}[1]{\textcolor[rgb]{0.31,0.60,0.02}{#1}}
\newcommand{\WarningTok}[1]{\textcolor[rgb]{0.56,0.35,0.01}{\textbf{\textit{#1}}}}
\usepackage{graphicx,grffile}
\makeatletter
\def\maxwidth{\ifdim\Gin@nat@width>\linewidth\linewidth\else\Gin@nat@width\fi}
\def\maxheight{\ifdim\Gin@nat@height>\textheight\textheight\else\Gin@nat@height\fi}
\makeatother
% Scale images if necessary, so that they will not overflow the page
% margins by default, and it is still possible to overwrite the defaults
% using explicit options in \includegraphics[width, height, ...]{}
\setkeys{Gin}{width=\maxwidth,height=\maxheight,keepaspectratio}
% Set default figure placement to htbp
\makeatletter
\def\fps@figure{htbp}
\makeatother
\setlength{\emergencystretch}{3em} % prevent overfull lines
\providecommand{\tightlist}{%
  \setlength{\itemsep}{0pt}\setlength{\parskip}{0pt}}
\setcounter{secnumdepth}{-\maxdimen} % remove section numbering

\title{STAT461 HW6}
\author{Xiangyu Ren}
\date{10/9/2020}

\begin{document}
\maketitle

\hypertarget{problem-1.-define-the-following-terms}{%
\paragraph{Problem 1. Define the following
terms:}\label{problem-1.-define-the-following-terms}}

\hypertarget{a.-treatment}{%
\paragraph{a. treatment}\label{a.-treatment}}

Combinations of factor levels are called treatments, where factor, also
called an independent variable, is an explanatory variable manipulated
by the experimenter.

\hypertarget{b.-contrast}{%
\paragraph{b. contrast}\label{b.-contrast}}

A contrast is a linear combination of variables such that all the
coefficients add up to zero. One way to think of it is as a set of
weighted variables.

\hypertarget{c.-replicate}{%
\paragraph{c.~replicate}\label{c.-replicate}}

Replication is the repetition of an experimental condition so that the
variability associated with the phenomenon can be estimated.

\hypertarget{d.-confounder}{%
\paragraph{d.~confounder}\label{d.-confounder}}

A confounder is a variable that influences both the dependent variable
and independent variable, causing a spurious association.

\hypertarget{e.-response}{%
\paragraph{e. response}\label{e.-response}}

A response variable, also known as a dependent variable, is a concept,
idea, or quantity that someone wants to measure.

\hypertarget{f.-generalizability}{%
\paragraph{f.~generalizability}\label{f.-generalizability}}

Generalizability is a measure of how well a researcher thinks their
experimental results from a sample can be extended to the population as
a whole.

\hypertarget{problem-2.}{%
\paragraph{Problem 2.}\label{problem-2.}}

\hypertarget{a.-enter-the-above-data-into-r.}{%
\paragraph{a. Enter the above data into
R.}\label{a.-enter-the-above-data-into-r.}}

\begin{Shaded}
\begin{Highlighting}[]
\NormalTok{unit =}\StringTok{ }\DecValTok{1}\OperatorTok{:}\DecValTok{10}
\NormalTok{brand =}\StringTok{ }\KeywordTok{c}\NormalTok{(}\StringTok{"A"}\NormalTok{, }\StringTok{"B"}\NormalTok{, }\StringTok{"A"}\NormalTok{, }\StringTok{"A"}\NormalTok{, }\StringTok{"B"}\NormalTok{, }\StringTok{"B"}\NormalTok{, }\StringTok{"B"}\NormalTok{, }\StringTok{"A"}\NormalTok{, }\StringTok{"A"}\NormalTok{, }\StringTok{"B"}\NormalTok{) }
\NormalTok{yield =}\StringTok{ }\KeywordTok{c}\NormalTok{(}\DecValTok{13}\NormalTok{, }\DecValTok{9}\NormalTok{, }\DecValTok{10}\NormalTok{, }\DecValTok{8}\NormalTok{, }\DecValTok{20}\NormalTok{, }\DecValTok{13}\NormalTok{, }\DecValTok{11}\NormalTok{, }\DecValTok{25}\NormalTok{, }\DecValTok{8}\NormalTok{, }\DecValTok{15}\NormalTok{)}
\NormalTok{data =}\StringTok{ }\KeywordTok{data.frame}\NormalTok{(unit, brand, yield)}
\NormalTok{data}
\end{Highlighting}
\end{Shaded}

\begin{verbatim}
##    unit brand yield
## 1     1     A    13
## 2     2     B     9
## 3     3     A    10
## 4     4     A     8
## 5     5     B    20
## 6     6     B    13
## 7     7     B    11
## 8     8     A    25
## 9     9     A     8
## 10   10     B    15
\end{verbatim}

\hypertarget{b.-run-the-one-way-anova-model-in-r.}{%
\paragraph{b. Run the one-way ANOVA model in
R.}\label{b.-run-the-one-way-anova-model-in-r.}}

\begin{Shaded}
\begin{Highlighting}[]
\KeywordTok{library}\NormalTok{(lsmeans)}
\end{Highlighting}
\end{Shaded}

\begin{verbatim}
## Loading required package: emmeans
\end{verbatim}

\begin{verbatim}
## The 'lsmeans' package is now basically a front end for 'emmeans'.
## Users are encouraged to switch the rest of the way.
## See help('transition') for more information, including how to
## convert old 'lsmeans' objects and scripts to work with 'emmeans'.
\end{verbatim}

\begin{Shaded}
\begin{Highlighting}[]
\NormalTok{model=}\KeywordTok{aov}\NormalTok{(yield}\OperatorTok{~}\NormalTok{brand,}\DataTypeTok{data=}\NormalTok{data)}
\KeywordTok{anova}\NormalTok{(model)}
\end{Highlighting}
\end{Shaded}

\begin{verbatim}
## Analysis of Variance Table
## 
## Response: yield
##           Df Sum Sq Mean Sq F value Pr(>F)
## brand      1    1.6    1.60  0.0467 0.8343
## Residuals  8  274.0   34.25
\end{verbatim}

\begin{Shaded}
\begin{Highlighting}[]
\NormalTok{lsm.data=}\KeywordTok{lsmeans}\NormalTok{(model, }\OperatorTok{~}\NormalTok{brand)}
\NormalTok{lsm.data}
\end{Highlighting}
\end{Shaded}

\begin{verbatim}
##  brand lsmean   SE df lower.CL upper.CL
##  A       12.8 2.62  8     6.76     18.8
##  B       13.6 2.62  8     7.56     19.6
## 
## Confidence level used: 0.95
\end{verbatim}

\hypertarget{c.-what-is-the-least-squares-estimate-for-the-contrast-ux3c4_aux3c4_b-compute-the-least-square-estimate-in-r.}{%
\paragraph{\texorpdfstring{c.~What is the least squares estimate for the
contrast \(τ_A−τ_B\)? Compute the least square estimate in
R.}{c.~What is the least squares estimate for the contrast τ\_A−τ\_B? Compute the least square estimate in R.}}\label{c.-what-is-the-least-squares-estimate-for-the-contrast-ux3c4_aux3c4_b-compute-the-least-square-estimate-in-r.}}

\end{document}
