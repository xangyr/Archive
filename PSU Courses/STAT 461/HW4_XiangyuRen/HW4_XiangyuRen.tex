% Options for packages loaded elsewhere
\PassOptionsToPackage{unicode}{hyperref}
\PassOptionsToPackage{hyphens}{url}
%
\documentclass[
]{article}
\usepackage{lmodern}
\usepackage{amssymb,amsmath}
\usepackage{ifxetex,ifluatex}
\ifnum 0\ifxetex 1\fi\ifluatex 1\fi=0 % if pdftex
  \usepackage[T1]{fontenc}
  \usepackage[utf8]{inputenc}
  \usepackage{textcomp} % provide euro and other symbols
\else % if luatex or xetex
  \usepackage{unicode-math}
  \defaultfontfeatures{Scale=MatchLowercase}
  \defaultfontfeatures[\rmfamily]{Ligatures=TeX,Scale=1}
\fi
% Use upquote if available, for straight quotes in verbatim environments
\IfFileExists{upquote.sty}{\usepackage{upquote}}{}
\IfFileExists{microtype.sty}{% use microtype if available
  \usepackage[]{microtype}
  \UseMicrotypeSet[protrusion]{basicmath} % disable protrusion for tt fonts
}{}
\makeatletter
\@ifundefined{KOMAClassName}{% if non-KOMA class
  \IfFileExists{parskip.sty}{%
    \usepackage{parskip}
  }{% else
    \setlength{\parindent}{0pt}
    \setlength{\parskip}{6pt plus 2pt minus 1pt}}
}{% if KOMA class
  \KOMAoptions{parskip=half}}
\makeatother
\usepackage{xcolor}
\IfFileExists{xurl.sty}{\usepackage{xurl}}{} % add URL line breaks if available
\IfFileExists{bookmark.sty}{\usepackage{bookmark}}{\usepackage{hyperref}}
\hypersetup{
  pdftitle={STAT461 HW4},
  pdfauthor={Xiangyu Ren},
  hidelinks,
  pdfcreator={LaTeX via pandoc}}
\urlstyle{same} % disable monospaced font for URLs
\usepackage[margin=1in]{geometry}
\usepackage{graphicx,grffile}
\makeatletter
\def\maxwidth{\ifdim\Gin@nat@width>\linewidth\linewidth\else\Gin@nat@width\fi}
\def\maxheight{\ifdim\Gin@nat@height>\textheight\textheight\else\Gin@nat@height\fi}
\makeatother
% Scale images if necessary, so that they will not overflow the page
% margins by default, and it is still possible to overwrite the defaults
% using explicit options in \includegraphics[width, height, ...]{}
\setkeys{Gin}{width=\maxwidth,height=\maxheight,keepaspectratio}
% Set default figure placement to htbp
\makeatletter
\def\fps@figure{htbp}
\makeatother
\setlength{\emergencystretch}{3em} % prevent overfull lines
\providecommand{\tightlist}{%
  \setlength{\itemsep}{0pt}\setlength{\parskip}{0pt}}
\setcounter{secnumdepth}{-\maxdimen} % remove section numbering

\title{STAT461 HW4}
\author{Xiangyu Ren}
\date{9/25/2020}

\begin{document}
\maketitle

\hypertarget{problem-1.-consider-a-completely-randomized-desgin-with-observations-on-three-treatments-coded-123.-for-the-one-way-anova-model-determine-which-of-the-following-are-estimable.-for-those-that-are-estimable-write-out-the-estimable-function-as-sum_i13-biux3bc-ux3c4_i-and-clearly-state-b1-b2-b3.-finally-for-those-that-are-estimable-state-the-least-squares-estimator.}{%
\paragraph{\texorpdfstring{Problem 1. Consider a completely randomized
desgin with observations on three treatments coded 1,2,3. For the
one-way ANOVA model, determine which of the following are estimable. For
those that are estimable, write out the estimable function as
\(\sum_{i=1}^3 bi(μ + τ_i)\) and clearly state b1, b2, b3. Finally, for
those that are estimable, state the least squares
estimator.}{Problem 1. Consider a completely randomized desgin with observations on three treatments coded 1,2,3. For the one-way ANOVA model, determine which of the following are estimable. For those that are estimable, write out the estimable function as \textbackslash sum\_\{i=1\}\^{}3 bi(μ + τ\_i) and clearly state b1, b2, b3. Finally, for those that are estimable, state the least squares estimator.}}\label{problem-1.-consider-a-completely-randomized-desgin-with-observations-on-three-treatments-coded-123.-for-the-one-way-anova-model-determine-which-of-the-following-are-estimable.-for-those-that-are-estimable-write-out-the-estimable-function-as-sum_i13-biux3bc-ux3c4_i-and-clearly-state-b1-b2-b3.-finally-for-those-that-are-estimable-state-the-least-squares-estimator.}}

\begin{enumerate}
\def\labelenumi{(\alph{enumi})}
\tightlist
\item
  \(τ_1+τ_2-2τ_3\)
  \[τ_1+τ_2-2τ_3 = 1*\bar Y_1+1*\bar Y_2-2*Y_3 = \sum_{i=1}^3 b_i(μ+τ_i)\]
\end{enumerate}

\end{document}
